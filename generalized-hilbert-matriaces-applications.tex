\documentclass[11pt]{article}

% PAGE -----------------------------------------------------------
\usepackage[autostyle]{csquotes}
\usepackage{geometry}
\usepackage{mathrsfs}
\geometry{letterpaper}
\geometry{margin=1in}
\usepackage{enumitem}
%\geometry{landscape}

% PACKAGES -------------------------------------------------------
\usepackage[english]{babel}
\usepackage{amsmath,amsthm,amssymb}
\usepackage{amsfonts}
\usepackage{xfrac} %For split-level fractions
\usepackage{tabu} %For flexible tables
\usepackage{array} %For complicated arrays
\usepackage{verbatim} %For \comment environment
\usepackage{graphicx} %For including graphs
\usepackage{subcaption}%For displaying multiple images in one row
\usepackage{listings}%For code blocks
\usepackage{hyperref}
\usepackage{indentfirst}
\usepackage{tikz}


% HEADERS & FOOTERS ----------------------------------------------
\usepackage{fancyhdr}
\pagestyle{fancy}
\renewcommand{\headrulewidth}{0pt} %decorative line
\lhead{}\chead{}\rhead{\thepage}
\lfoot{}\cfoot{}\rfoot{}

% SECTION TITLE APPEARANCE
\usepackage{sectsty}


% THEOREMS -------------------------------------------------------
\newtheorem{thm}{Theorem}[section]
\newtheorem{cor}[thm]{Corollary}
\newtheorem{lem}[thm]{Lemma}
\newtheorem{prop}[thm]{Proposition}
\theoremstyle{definition}
\newtheorem{defn}[thm]{Definition}
\theoremstyle{remark}
\newtheorem{rem}[thm]{Remark}
\newtheorem{ex}{Example}[section]
\numberwithin{equation}{section}
% ----------------------------------------------------------------

%Just some declaremathoperators
\DeclareMathOperator{\Rad}{Rad}
\DeclareMathOperator{\BetaDistr}{Beta}
\DeclareMathOperator{\Bern}{Bern}
\DeclareMathOperator{\mode}{mode}
\DeclareMathOperator{\var}{var}
\DeclareMathOperator{\cov}{cov}
\DeclareMathOperator{\Dir}{Dir}
\DeclareMathOperator{\Tr}{Tr}
\DeclareMathOperator{\End}{End}
\DeclareMathOperator{\ad}{ad}
\DeclareMathOperator{\argmin}{argmin}
\DeclareMathOperator{\characteristic}{char}
\begin{document}
\section{Introduction}

\begin{defn} 
Let $\mathbb{Z} \ni p \ge 0$. A \emph{generalized Hilbert matrix} is a matrix $H\in M_n(\mathbb{R})$ whose entries are defined: 
$H_{ij} = \frac{1}{i+j-1+p}$
\end{defn}
Our problem is to find $a\in\mathbb{R}^N$ such that 
\begin{equation}
a^T H a\end{equation}
is minimized, where $H$ is a generalized Hilbert matrix (in our case, p=2), subject to the constraint: 

\begin{equation}
B^T a = \begin{pmatrix} 1 \\ -1\end{pmatrix}
\end{equation}
where \[B = \begin{pmatrix} 1 & 0\\ 
0 & 1 \\
\vdots & \vdots
\\ 
0 & 1 \end{pmatrix} \in M_{N\times 2}(\mathbb{R})\]

Our goal is to prove that such an $a$ has coefficients that alternate in sign. Precisely: 

\begin{thm}
Let $H$ be a generalized Hilbert matrix of size $N$ and $B \in M_{N\times 2}$ is as above. Furthermore, let 
\[ \hat{a} = \argmin_{a\in\mathbb{R}^N} \quad \quad a^THa : \quad B^Ta = \begin{pmatrix} 1 \\ -1 \end{pmatrix}\]
Then for odd $i$, $\hat{a}(i) > 0$ and for even $i$ $\hat{a}(i) < 0$.
\end{thm} 

We can obtain expressions for $\hat{a}$ via the method of Lagrange multipliers. Our problem amounts to minimizing 
\[ \frac{1}{2} a^T H a - \begin{pmatrix} \lambda_1 \\ \lambda_2 \end{pmatrix} (B^Ta - \begin{pmatrix} 1 \\ -1 \end{pmatrix} )\] 
Taking the derivative with respect to $a$, we see that we must have 
\[ a = H^{-1} B \begin{pmatrix} \lambda_1 \\ \lambda_2 \end{pmatrix} = H^{-1} \begin{pmatrix} \lambda_1 \\ \lambda_2 \\ \vdots \\ \lambda_2 \end{pmatrix}\]
and the constraint of $B^Ta$ results in :

\begin{align*}
B^T H^{-1} B \begin{pmatrix}\lambda_1 \\ \lambda_2 \end{pmatrix} = \begin{pmatrix} 1 \\ -1 \end{pmatrix}\\
&\implies \begin{pmatrix} \lambda_1 \\ \lambda_2 \end{pmatrix} = \frac{1}{}
\end{align*}

\end{document}