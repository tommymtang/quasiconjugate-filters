\documentclass{article}


\usepackage{arxiv} % for arxiv template format a

\usepackage[utf8]{inputenc} % allow utf-8 input
\usepackage[T1]{fontenc}    % use 8-bit T1 fonts
\usepackage{hyperref}       % hyperlinks
\usepackage{url}            % simple URL typesetting
\usepackage{booktabs}       % professional-quality tables
\usepackage{amsfonts}       % blackboard math symbols
\usepackage{nicefrac}       % compact symbols for 1/2, etc.
\usepackage{microtype}      % microtypography
\usepackage{lipsum}		% Can be removed after putting your text content

\usepackage[english]{babel}  % english
\usepackage{amsmath,amsthm,amssymb}% for theorems
% THEOREMS -------------------------------------------------------
\newtheorem{thm}{Theorem}[section]
\newtheorem{cor}[thm]{Corollary}
\newtheorem{lem}[thm]{Lemma}
\newtheorem{prop}[thm]{Proposition}
\theoremstyle{definition}
\newtheorem{defn}[thm]{Definition}
\theoremstyle{remark}
\newtheorem{rem}[thm]{Remark}
\newtheorem{ex}{Example}[section]
\numberwithin{equation}{section}
% ----------------------------------------------------------------

%Just some declaremathoperators
\DeclareMathOperator{\Rad}{Rad}
\DeclareMathOperator{\BetaDistr}{Beta}
\DeclareMathOperator{\Bern}{Bern}
\DeclareMathOperator{\mode}{mode}
\DeclareMathOperator{\var}{var}
\DeclareMathOperator{\cov}{cov}
\DeclareMathOperator{\Dir}{Dir}
\DeclareMathOperator{\Tr}{Tr}
\DeclareMathOperator{\End}{End}
\DeclareMathOperator{\ad}{ad}
\DeclareMathOperator{\argmin}{argmin}
\DeclareMathOperator{\characteristic}{char}


\title{On minimizing odd-even polynomial pair differences to construct steerable near-conjugate filter pairs}

%\date{September 9, 1985}	% Here you can change the date presented in the paper title
%\date{} 					% Or removing it

\author{
 Tommy M. Tang  \\
  Department of Radiology and Biomedical Imaging\\
  Yale University\\
   300 Cedar St, New Haven\\
  New Haven, 06510 \\
  \texttt{tommymtang@gmail.com} \\
  %% examples of more authors
   \And
 Hemant D. Tagare \thanks{ Department of Biomedical Engineering, Electrical Engineering, Statistics and Data Science} \\
 Department of Radiology and Biomedical Imaging\\
 Yale University\\
 300 Cedar St, New Haven\\
  New Haven, 06510 \\
  \texttt{hemant.tagare@yale.edu} \\
  %% \AND
  %% Coauthor \\
  %% Affiliation \\
  %% Address \\
  %% \texttt{email} \\
  %% \And
  %% Coauthor \\
  %% Affiliation \\
  %% Address \\
  %% \texttt{email} \\
  %% \And
  %% Coauthor \\
  %% Affiliation \\
  %% Address \\
  %% \texttt{email} \\
}

% Uncomment to remove the date
%\date{}

% Uncomment to override  the `A preprint' in the header
%\renewcommand{\headeright}{Technical Report}
%\renewcommand{\undertitle}{Technical Report}

\begin{document}
\maketitle

\begin{abstract}
For the construction of axially symmetric three-dimensional filters pairs to approximately represent the real and imaginary parts of a quadrature filter - giving the analytic representation of the impulse response of a filter - it is necessary to construct a pair of polynomials - one odd, one even - that differ minimally on the unit interval. This minimization is convex and therefore straightforward to implement. However, it is not clear that such a pair of polynomials would necessarily exhibit other desirable properties of this minimization, such as positivity, concentration at one, or that this difference would vanish as we allow our maximum bandwidth N to increase. We demonstrate that under the constraints that the polynomials vanish at the origin and are normalized so that the coefficients add up to one, we may in fact guarantee (1) the optimally similar polynomial pairs have positive coefficients, (2) the objective function loss vanishes as the maximum degree N allowed approaches infinity and (3) these polynomials converge pointwise to the indicator function at 1 on the relevant unit interval. In the process, we explore various properties of the closely related Hilbert matrices and their inversion. 
\end{abstract}


% keywords can be removed
\keywords{steerability \and quadrature \and filter \and quadratic programming \and hilbert matrices}


\section{Introduction}
In this section we give the application motives for our results. We introduce the need to approximately represent quadrature filters using near-conjugate filters and explore briefly certain sufficient conditions for steerability. We precisely define our optimization problem in terms of an odd and even polynomial pair under a bandwidth N, and finally reformulate our problem in vectorized form using generalized Hilbert matrices, which is the line of thinking we will adopt for the rest of the paper. 
\subsection{Motivation}
On minimizing conjugacy loss of odd and even polynomial filter pairs to represent steerable near quadrature filters
On steerable near-conjugate polynomial filters pairs to approximate quadrature filters
On minimizing odd-even polynomial pair differences to construct steerable near-conjugate filter pairs
[GIVE BACKGROUND ON QUADRATURE FILTERS, NEAR GABOR REPRESENTATIONS, STEERABILITY.... MOTIVATE THE NEED TO FIND SUCH POLYNOMIAL PAIRS]
Quadrature filters are analytic representations of the impulse response of a real-valued filter - such as a Gabor filter. Since nonzero quadrature filters are complex, they are typically represented as two real-valued filters corresponding to the real and imaginary part of the filter. Therefore it is desirable to construct a pair of functions - one even, one odd - respectively, to represent the filter. In many applications of such filters, it is computationally desirable that such functions are exactly steerable. However, this requires that 

\subsection{The quadratic optimization problem}
\label{problem}

We now formulate the problem precisely. 

\subsection{Reformulation with Generalized Hilbert Matrix}
\label{reformulation-hilbert}
\begin{defn}\label{generalized-hilbert-matrix}
A \emph{generalized Hilbert matrix with order p} 
\end{defn}

\paragraph{Paragraph}
\lipsum[7]
\section{Important Properties of the Generalized Hilbert Matrix}

\label{sec:others}
\lipsum[8] \cite{kour2014real,kour2014fast} and see \cite{hadash2018estimate}.

The documentation for \verb+natbib+ may be found at
\begin{center}
  \url{http://mirrors.ctan.org/macros/latex/contrib/natbib/natnotes.pdf}
\end{center}
Of note is the command \verb+\citet+, which produces citations
appropriate for use in inline text.  For example,
\begin{verbatim}
   \citet{hasselmo} investigated\dots
\end{verbatim}
produces
\begin{quote}
  Hasselmo, et al.\ (1995) investigated\dots
\end{quote}

\begin{center}
  \url{https://www.ctan.org/pkg/booktabs}
\end{center}


\subsection{Definitions and Notations}
\lipsum[10] 
See Figure \ref{fig:fig1}. Here is how you add footnotes. \footnote{Sample of the first footnote.}
\lipsum[11] 

\begin{figure}
  \centering
  \fbox{\rule[-.5cm]{4cm}{4cm} \rule[-.5cm]{4cm}{0cm}}
  \caption{Sample figure caption.}
  \label{fig:fig1}
\end{figure}

\subsection{Exact expression for entries and sum of entries}
\lipsum[12]
See awesome Table~\ref{tab:table}.

\begin{table}
 \caption{Sample table title}
  \centering
  \begin{tabular}{lll}
    \toprule
    \multicolumn{2}{c}{Part}                   \\
    \cmidrule(r){1-2}
    Name     & Description     & Size ($\mu$m) \\
    \midrule
    Dendrite & Input terminal  & $\sim$100     \\
    Axon     & Output terminal & $\sim$10      \\
    Soma     & Cell body       & up to $10^6$  \\
    \bottomrule
  \end{tabular}
  \label{tab:table}
\end{table}

\subsection{Notes on magnitudes and signs of important sums}
\begin{itemize}
\item Lorem ipsum dolor sit amet
\item consectetur adipiscing elit. 
\item Aliquam dignissim blandit est, in dictum tortor gravida eget. In ac rutrum magna.
\end{itemize}

\section{Theorems and Results}
\subsection{The positivity of optimal polynomial pair coefficients}

\subsection{Asymptotic behavior of polynomial pairs}
\bibliographystyle{unsrt}  
%\bibliography{references}  %%% Remove comment to use the external .bib file (using bibtex).
%%% and comment out the ``thebibliography'' section.


%%% Comment out this section when you \bibliography{references} is enabled.
\begin{thebibliography}{1}

\bibitem{kour2014real}
George Kour and Raid Saabne.
\newblock Real-time segmentation of on-line handwritten arabic script.
\newblock In {\em Frontiers in Handwriting Recognition (ICFHR), 2014 14th
  International Conference on}, pages 417--422. IEEE, 2014.

\bibitem{kour2014fast}
George Kour and Raid Saabne.
\newblock Fast classification of handwritten on-line arabic characters.
\newblock In {\em Soft Computing and Pattern Recognition (SoCPaR), 2014 6th
  International Conference of}, pages 312--318. IEEE, 2014.

\bibitem{hadash2018estimate}
Guy Hadash, Einat Kermany, Boaz Carmeli, Ofer Lavi, George Kour, and Alon
  Jacovi.
\newblock Estimate and replace: A novel approach to integrating deep neural
  networks with existing applications.
\newblock {\em arXiv preprint arXiv:1804.09028}, 2018.

\end{thebibliography}


\end{document}